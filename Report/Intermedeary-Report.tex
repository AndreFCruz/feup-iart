\documentclass[a4paper]{article}

%use the english line for english reports
%usepackage[english]{babel}
\usepackage[portuguese]{babel}
\usepackage[utf8]{inputenc}
\usepackage{indentfirst}
\usepackage{graphicx}
\usepackage{verbatim}
\usepackage{fancyhdr}
\usepackage{listings}
\usepackage{color}

\definecolor{dkgreen}{rgb}{0,0.6,0}
\definecolor{gray}{rgb}{0.5,0.5,0.5}
\definecolor{mauve}{rgb}{0.58,0,0.82}

\lstset{frame=tb,
  language=Prolog,
  aboveskip=3mm,
  belowskip=3mm,
  showstringspaces=false,
  columns=flexible,
  basicstyle={\small\ttfamily},
  numbers=none,
  numberstyle=\tiny\color{gray},
  keywordstyle=\color{blue},
  commentstyle=\color{dkgreen},
  stringstyle=\color{mauve},
  breaklines=true,
  breakatwhitespace=true,
  tabsize=3
  }

\begin{document}


\setlength{\textwidth}{16cm}
\setlength{\textheight}{22cm}
\title{
\includegraphics[scale=0.6]{images/feup-logo.png}\linebreak\linebreak\linebreak
\Huge\textbf{Redes Neuronais para a identificação de Pulsars}\linebreak\linebreak\linebreak
\Large\textbf{Relatório Intercalar}\linebreak\linebreak
\Large{Inteligência Artificial}\linebreak\linebreak\linebreak\linebreak\linebreak\linebreak
\Large{3º ano do Mestrado Integrado em Engenharia Informática e Computação}
}

\author{\textbf{Grupo E1\_3:}\\
\linebreak\\
André Cruz - 201503776 \\
Edgar Carneiro - 201503748 \\
João Carvalho - 201504875 \\
\vspace{1cm}}

\maketitle
\thispagestyle{empty}

%************************************************************************************************
%************************************************************************************************

\tableofcontents

\newpage

%************************************************************************************************
%************************************************************************************************

\section{Objetivo}

\newpage

%*************************************************************************************************
%*************************************************************************************************

\section{Descrição}

\subsection{Especificação}

\subsubsection{Descrição e análise do dataset. Pré-processamento dos dados.}

\subsubsection{Modelo de aprendizagem a aplicar: redes neuronais.}

\subsubsection{Redes neuronais: arquitectura, configuração prevista da rede.}


\subsection{Trabalho Efetuado}

\subsection{Resultados esperados e forma de avaliação}

\newpage

%*************************************************************************************************
%*************************************************************************************************

\section{Conclusões}

\newpage

%*************************************************************************************************
%*************************************************************************************************

\section{Recursos}

\newpage

\end{document}
